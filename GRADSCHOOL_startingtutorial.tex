% Programming Stuff
Download \textbf{Sublime Text} 
https://www.sublimetext.com/3
Once it is installed:
First and foremost: Install: \textbf{Package Control}
- Allows you to install/upgrade/remove/view all packages and plugins that you have installed (super easy to call up)
- https://packagecontrol.io/installation
- Restart \textbf{Sublime}
Then using the Command Palette (shift+command+P)
Just start typing \textbf{Package Control} and select the \textbf{Install Package} options
Then start typing the following and select when the one you want appears
— \textbf{Latex Tools}
— \textbf{Latexing}
— \textbf{Latex-cwl} (autocomplete latex commands)
— \textbf{Git}
— \textbf{Git Tools}
— \textbf{Shell Exec} (executing shell commands)

% To make Register your version of Sublime: 
% 1. Go to http://appnee.com/sublime-text-3-universal-license-keys-collection-for-win-mac-linux/
% 		- Copy a License (for recently downloaded Sublime, choose one "After 309")
% 2. In Sublime:
% 		- Help > Input License
% 		- Paste into Input License Box
% 		- Remove the last couple lines (after "Copyright")
% 3. You're done! 

% Next: Need to add Git HUb connection and making ssh keys instructions
When you're connecting an already created Repository on GitHub to your computer, you need to create an SSH key
One your computer:
- In your: /Users/you/.ssh directory you want to create a public key in your id_rsa folders
- You may already have one in id_rsa.pub
- Copy this \textit{public key}
On GitHub:
Go to: Setting > SSH and GPG Keys
- Copy+paste \textit{public key} that into a \textbf{new SSH key}
Back to Computer:
- Now you need to clone your new directory
- In terminal: Go to your repositories directory and determine where you want to have your cloned directory
- Then \$ git clone git@github.com:yourgithubscreenname/yourrepositoryname.git
- You'll have a new file under your repositories name
- Push/Commit/Pull/Enjoy!


% Citation and Writing Stuff
Download \textbf{Mendeley}:
https://www.mendeley.com/download-mendeley-desktop/mac/instructions/
Once it is installed: 
In your \textbf{Documents} folder create a Mendeley directory
Inside that \textbf{Mendeley directory} create three others:
- bibtexfiles (where all your bibtext files will be created and saved)
- tempfiles   (where you will save your papers straight off the web, NO NEED TO EDIT FILE NAMES even if files saves as ``pdf.pdf'')
- files       (where your ``pdf.pdf'' awkwardly named files will be resaved with awesome, accurate names)

Now open \textbf{Mendeley}:
Go to \textit{Mendeley Desktop} > \textit{Preferences} 
- Under \textbf{Document Details}:
	Make sure ``Citation Key'' and ``DOI'' are selected
- Under \textbf{File Organizer}:
	Select \textbf{Organize My Files}
		In the \textbf{Copy files to} area put in the path to your Mendeley/files directory
		For example: /Users/the_neekster/Documents/MENDELEY/MEND_files
	Select \textbf{Sort files into subfolders}
		You can adjust this to your preferences
		I use: Year, Title, and Journal
		This creates a subdirectory in your /Mendeley/files directory
		For example: ~/MENDELEY/MEND_files/Treister_et_al
	Select \textbf{Rename document files}
		Also adjust to your preference
		This renames your files nicely
- Under \textbf{Watched Folders}:
	Find your Mendeley folder and select it (so that all three folders inside are also selected)
- Under \textbf{BibTeX}:
	Select \textbf{Enable BibTeX Syncing}
	Select \textbf{Create one BibTeX file per group}
		In the \textbf{Path} area put in the path to your Mendeley/bibtexfiles directory
		For example: /Users/the_neekster/Documents/MENDELEY/MEND_bibtexfiles

Yay! You're ready to get started!
How to add a new paper citation:
- Create a folder, say for a paper 
	Everything you put in this folder will be in a single bibtex file
	For example: \textit{Sanchez2016}
- Open your paper on your favorite web browser and save to your /Mendeley/bibtexfile
	DON'T WORRY ABOUT CHANGING THE NAME OF THE FILE
- Open Mendeley and watch it upload 
	** If this doesn't work try restarting Mendeley
- Go to the \textbf{All Documents} or \textbf{Unsorted}
	Add the new file to the \textit{Sanchez2016} folder you made 
	It should immediately create or add the paper to the bibtexfile
- Add your bibliography to the end of your LaTeX Document:
	For example: \bibliography{/Users/nicolesanchez/Documents/MENDELEY/MEND_bibtexfiles/Sanchez2016.bib}
	If you use this inconjunction with Sublime, it will automatically update and offer you paper suggestions from within that file! So handy! 
You're done! Go add more papers!

